\documentclass[a4paper, 12pt]{article}

\usepackage[portuges]{babel}
\usepackage[utf8]{inputenc}
\usepackage{amsmath}
\usepackage{indentfirst}
\usepackage{graphicx}
%\begin{figure}[!ht]
%\centering
%\includegraphics[width=2cm]{c:/ufba.jpg}
%\end{figure}
%Imagens dentro do ambiente multicols:
%\begingroup
%\centering
%\includegraphics[width=10cm]{h:/Will/Trabalhos_2019/tabela_1.jpg}
%\endgroup
\usepackage[a4paper,left=1.5cm,right=1.5cm,top=2.5cm,bottom=2cm]{geometry}
\usepackage{multicol}
%\begin{multicols}{2}
%\end{multicols}
\usepackage{amsfonts}
\usepackage{amssymb}
\usepackage{lipsum}


\begin{document}
\begin{center}
\large\textbf{Relatório dos experimentos sobre circuito RC e circuito RL}\\
\end{center}
\begin{center}
\small
$^{1}$ Rafael Mendonça Borges, $^{2}$ Walter Francisco dos Santos Filho, $^{3}$ William Makoto Hakamada \\
\end{center}
\begin{center}
\small
$^{123}$ Departamento de Física, Universidade Estadual de Londrina, 86.057-970, Londrina, Paraná, Brasil. \\

\end{center}
\begin{center}
\textit{26 de setembro de 2019}
\end{center}
        \begin{figure}[!ht]
        \centering
        \includegraphics[width=16cm]{h:/Will/Trabalhos/linha_preta.jpg}
        \end{figure}
\noindent Este relatório busca encontrar e observar relações em circuitos de corrente alternada quando na presença de dispositivos eletrônicos que garantem a existência de um certa reatância. Dessa forma, três práticas foram realizadas coletando dados de tensão e diferença de tempo entre um resistor de teste e um dispositivo de prova em diferentes frequências para um resistor, um capacitor e um indutor. Os resultados encontrados foram razoáveis e diferem levemente do esperado medido diretamente, levando à hipótese de uma limitação instrumental. Ademais, os valores característicos obtidos de cada prática pelo método gráfico foram: $R = (73,28\pm 0,99) \Omega$ , $C = (11,95\pm 0,11) \mu F$ e $L = (20,30\pm 1,33) mH$.
        \begin{figure}[!ht]
        \centering
        \includegraphics[width=16cm]{h:/Will/Trabalhos/linha_preta.jpg}
        \end{figure}
\begin{multicols}{2}
\section{Introdução}
\paragraph{} De fato, ainda nos dias de hoje, o rádio é muito presente na vida das pessoas que o usa como fonte de informação e entretenimento. Além disso, o rádio também fora, outrora, o principal meio de comunicação em tempos de guerra, a saber, na Segunda Guerra Mundial. Dada tamanha importância, uma de suas principais propriedades é a forma de distinguir e selecionar faixas de frequências desejáveis feitas através de filtros.

\paragraph{} Os filtros citados acima estão intimamente ligados à reatância de um circuito de corrente alternada determinada através de característica de elementos do circuito e da frequência angular de oscilação estabelecida pela fonte, itens importantíssimos durante a sincronização.  

\paragraph{} A fim de notar as relações existentes entre a reatância e propriedades do circuito, estará incluso no relatório uma abordagem teórica buscando estabelecer algumas definições do que será estudado, seguido de uma análise prática em busca de alguns parâmetros a serem comparados com a medição direta. 

\section{Métodos}
\subsection{Modelo Teórico}
\paragraph{} É denominado um circuito de corrente alternada qualquer circuito que esteja submetido a uma fonte cuja diferença de potencial varie conforme uma função senoidal do tempo:

\begin{equation}
V = V_0cos(\omega t).
\end{equation}

\paragraph{} Onde $V$ é a ddp em um tempo t, $V_0$ é a máxima diferença de potencial, ou amplitude, e $\omega$ é a frequência angular de oscilação.

\paragraph{} Um tal circuito é percorrido por uma corrente elétrica que também varia senoidalmente conforme a seguinte equação:

\begin{equation}
i=i_0cos(\omega t).
\end{equation}

\paragraph{} Onde $i_0$ se trata da corrente máxima.

\paragraph{} Considere inicialmente um circuito constituído por uma fonte de corrente alternada e um resistor. Levando em conta a equação (2) e a definição da resistência R, temos que a ddp no resistor é dada por:

\begin{equation}
V_R = Ri \rightarrow V_R = Ri_0cos(\omega t)
\end{equation}
\begin{equation}
V_R = V_{R_0}cos(\omega t)
\end{equation}

\paragraph{} Onde $V_{R_0}$ é a ddp máxima no resistor.

\paragraph{} Dando um passo adiante, considere agora um capacitor de capacitância $C$ no lugar do resistor. Uma vez que a ddp no capacitor $V_c$ depende de sua carga acumulada, uma boa proposta é usar sua definição e desenvolver algebricamente o conceito de carga para obter uma expressão conveniente da seguinte maneira:

\begin{equation}
V_c = \frac{q}{C} \rightarrow V_c = \frac{1}{C} \int \frac{dq}{dt} dt \rightarrow 
\nonumber
\end{equation}
\begin{equation}
V_c = \frac{1}{C} \int_0 ^t i_0cos(\omega t)dt \rightarrow V_c = \frac{1}{\omega C} i_0sen(\omega t) 
\nonumber
\end{equation}
\begin{equation}
\rightarrow V_c = \frac{1}{\omega C}i_0cos(\omega t - \frac{\pi}{2}).
\end{equation}

\paragraph{} Observe que a ddp máxima do capacitor será dada por:

\begin{equation}
V_{c_0} = \frac{i_0}{\omega C}.
\end{equation}

\paragraph{} Então, a partir de (5) e (6) é possível escrever:

\begin{equation}
Vc = V_{c_o} cos(\omega t - \frac{\pi}{2})
\end{equation}

\paragraph{} Duas conclusões podem ser feitas com base em (7). A primeira delas é o atraso de fase de $V_c$ em relação a $V$ que é de $\frac{\pi}{2} rad$. A segunda conclusão é uma analogia da equação (5) com a equação (3), sendo conveniente chamar o termo $X_c = \frac{1}{\omega C}$ de reatância capacitiva um vez que ele se comporta como um resistor no circuito capacitivo.

\paragraph{} Fazendo o mesmo processo para um circuito indutivo usando o conceito de fem auto-induzida, temos:

\begin{equation}
V_L = L\frac{di}{dt} \rightarrow V_L = L \frac{d}{dt}[i_0cos(\omega t)] \rightarrow 
\nonumber
\end{equation}
\begin{equation}
V_L = -\omega L i_0sen(\omega t)
\nonumber
\end{equation}
\begin{equation}
\rightarrow V_L = \omega Li_0cos(\omega t + \frac{\pi}{2}).
\end{equation}
\begin{equation}
V_L = V_{L_0}cos(\omega t + \frac{\pi}{2}).
\end{equation}

\paragraph{} Em que $V_{L_0}$ é a ddp máxima no indutor.

\paragraph{} Sendo assim, a equação (9) nos indica que a fase de $V_L$ está adiantada $\frac{\pi}{2} rad$ em relação a $V$ e, também fazendo analogia com a equação (3), a partir de (8) definimos $X_L = \omega L$ como sendo a reatância indutiva do circuito. [1]

\subsection{Métodos Experimentais}
Para os experimentos de circuito resistivo, capacivo e indutivo, todos em corrente alternada, foram utilizados os equipamentos mostrados na figura 1:
\begin{center}
\begingroup
    \centering
    \includegraphics[width=5cm]{h:/Will/Trabalhos_2019/esquema_circuito.jpeg}
\endgroup
\end{center}
Figura 1: Foto dos equipamentos usados nos experimentos, onde para a realização, foi necessário um gerador de frequência, que tem como objetivo alterar a frequência de oscilação; um capacitor; indutor; resistência e um osciloscópio que mostra em sua tela a voltagem da onda no eixo y, e o tempo que passa a corrente no eixo x. \\

Esses equipamentos estão ligados um ao outro, para que formem um circuito resistivo, capacitivo ou indutivo, cujo capacitor na teoria tem um valor de capacitância de $15 \mu F$, o indutor uma indutância de $20 mH$ e a resistência $100 \Omega$. O gedador teve sua frequência variada de $200 Hz$ até $2000 Hz$ em intervalos de $200 Hz$ e seu erro foi aproximado para 0,5\% do valor da leitura. Assim com o osciloscópio mediu-se a tensão pico a pico entre os terminais do resistor e depois entre os terminais do capacitor ou indutor, afim de se mensurar a reatância capacitiva ou indutiva rescpectivamente, depedendo da prática que está sendo realizada.
\section{Resultados e Discussão}
Por uma dificuldade encontrada com os equipamentos, os dados para as práticas sobre curcuito resistivo e capacitivo foram adquiridos em conjunto com outras bancadas do laboratório.
\subsection{Circuito Resistivo}
Dados Coletados:
\begin{center}
\begingroup
    \centering
    \includegraphics[width=9cm]{h:/Will/Trabalhos_2019/tabela_1.jpg}
\endgroup
\end{center}
Tabela 1: Relação entre as frequências, controladas pelo gerador de frequências; as tensões, mensuradas pelo osciloscópio; e a diferença de tempo entre as ondas, também mensurado pelo osciloscópio; onde $V_{rt}$ é a tensão no resistor teste de $10 \Omega$ e $V_{rp}$ é a tensão no resistor de interesse. \\

A partir da análise desses dados, temos:
\begin{center}
\begingroup
    \centering
    \includegraphics[width=9cm]{h:/Will/Trabalhos_2019/tabela_2.jpg}
\endgroup
\end{center}
Tabela 2: Dados inferidos a partir da análise dos dados coletados; onde $i$ é a corrente elétrica dada por: $i=V_{rt}/R_{t}$ e $R_{t}=10\Omega$, $R$ é a resistência do resistor de interesse dada por: $R=V_{rp}/i$, $\omega$ é a frequência angular dada por: $\omega=2.\pi . f$, e $\Delta \varphi$ é a diferença de fase entre as ondas dada por: $\Delta \varphi=\omega . \Delta t$; cada um com sua respectiva propagação de erro associada. \\

Construindo um gráfico $R(\Omega)$ X $\omega(Hz)$ podemos verificar a dependência da resistência com a frequência angular.
\begin{center}
\begingroup
    \centering
    \includegraphics[width=9cm]{h:/Will/Trabalhos_2019/grafico_resistencia.jpg}
\endgroup
\end{center}
Figura 2: Gráfico da resistência ($R$) em função da frequência angular ($\omega$), para o circuito resistivo. \\

Como apresentado no modelo teórico, a resistência não deveria depender da frequência angular, portanto ignorando os dois primeiros pontos e traçando a regreção linear, o coeficiente linear da reta, é equivalente ao valor da resistência do resistor de interesse. Portanto, o valor da resistência pelo método gráfico é de $73,28 \pm 0,99 \Omega$.

\subsection{Circuito Capacitivo}
Dados Coletados:
\begin{center}
\begingroup
    \centering
    \includegraphics[width=9cm]{h:/Will/Trabalhos_2019/tabela_3.jpg}
\endgroup
\end{center}
Tabela 3: Relação entre as frequências, controladas pelo gerador de frequências; as tensões, mensuradas pelo osciloscópio; e a diferença de tempo entre as ondas, também mensurado pelo osciloscópio; onde $V_{rt}$ é a tensão no resistor teste de $10 \Omega$ e $V_{c}$ é a tensão no capacitor. \\

A partir da análise desses dados, temos:
\begin{center}
\begingroup
    \centering
    \includegraphics[width=8cm]{h:/Will/Trabalhos_2019/tabela_4.jpg}
\endgroup
\end{center}
Tabela 4: Dados inferidos a partir da análise dos dados coletados; onde $i$ é a corrente elétrica dada por: $i=V_{rt}/R_{t}$ e $R_{t}=10\Omega$, $X_{C}$ é a reatância capacitiva dada por: $X_{C}=V_{c}/i$, $\omega$ é a frequência angular dada por: $\omega=2.\pi . f$, e $\Delta \varphi$ é a diferença de fase entre as ondas dada por: $\Delta \varphi=\omega . \Delta t$; cada um com sua respectiva propagação de erro associada. \\

Construindo um gráfico $X_{C}(\Omega)$ X $[1/ \omega] (s)$ podemos verificar a dependência da reatância capacitiva com a frequência angular.
\begin{center}
\begingroup
    \centering
    \includegraphics[width=9cm]{h:/Will/Trabalhos_2019/grafico_reatancia_cap.jpg}
\endgroup
\end{center}
Figura 3: Gráfico da reatância capacitiva ($X_{C}$) em função de ($1/ \omega$), para o circuito capacitivo. \\

Como apresentado no modelo teórico, a reatância capacitiva deveria depender da frequência angular de acordo com a relação:
\begin{center}
$X_{C}=\dfrac{1}{\omega .C}$
\end{center} 
onde $C$ é a capacitância do capacitor. Traçando a regreção linear, o coeficiente angular da reta, é equivalente ao valor de $1/C$, pois:
\begin{center}
$C=\dfrac{1}{\omega .X_{C}}$ e $b=\omega .X_{C}$
\end{center} 
onde $b$ é o coeficiente angular da reta de ajuste. Portanto, o valor da capacitância do capacitor pelo método gráfico é de $(11,95 \pm 0,11) \mu F$.

\subsection{Circuito Indutivo}
Dados Coletados:
\begin{center}
\begingroup
    \centering
    \includegraphics[width=8cm]{h:/Will/Trabalhos_2019/tabela_5.jpg}
\endgroup
\end{center}
Tabela 5: Relação entre as frequências, controladas pelo gerador de frequências; as tensões, mensuradas pelo osciloscópio; e a diferença de tempo entre as ondas, também mensurado pelo osciloscópio; onde $V_{rt}$ é a tensão no resistor teste de $10 \Omega$ e $V_{L}$ é a tensão no resistor de interesse. \\

A partir da análise desses dados, temos:
\begin{center}
\begingroup
    \centering
    \includegraphics[width=8cm]{h:/Will/Trabalhos_2019/tabela_6.jpg}
\endgroup
\end{center}
Tabela 6: Dados inferidos a partir da análise dos dados coletados; onde $i$ é a corrente elétrica dada por: $i=V_{rt}/R_{t}$ e $R_{t}=10\Omega$, $X_{L}$ é a reatância indutiva dada por: $X_{L}=V_{L}/i$, $\omega$ é a frequência angular dada por: $\omega=2.\pi . f$, e $\Delta \varphi$ é a diferença de fase entre as ondas dada por: $\Delta \varphi=\omega . \Delta t$; cada um com sua respectiva propagação de erro associada. \\

Construindo um gráfico $X_{L}(\Omega)$ X $\omega(Hz)$ podemos verificar a dependência da resistência com a frequência angular.
\begin{center}
\begingroup
    \centering
    \includegraphics[width=9cm]{h:/Will/Trabalhos_2019/grafico_reatancia_ind.jpg}
\endgroup
\end{center}
Figura 4: Gráfico da reatância indutiva ($\Omega$) em função da frequência angular ($\omega$), para o circuito indutivo. \\

Como apresentado no modelo teórico, a reatância indutiva deveria depender da frequência angular de acordo com a relação:
\begin{center}
$X_{L}=\omega .L$
\end{center} 
onde $L$ é a indutância do indutor. Traçando a regreção linear, o coeficiente angular da reta, é equivalente ao valor de $L$, pois:
\begin{center}
$L=\dfrac{X_{L}}{\omega}$ e $b=\dfrac{X_{L}}{\omega}$
\end{center} 
onde $b$ é o coeficiente angular da reta de ajuste. Portanto, o valor da indutância do indutor pelo método gráfico é de $(20,30 \pm 1,33) mH$.
\section{Conclusão}
Foi concluido que os valores teóricos estão próximos dos experimentais, sendo a capacitância teórica de $10 \mu F$ e a experimental, encontrada através do método gráfico, foi de $(11,95 \pm 0,11) \mu F$. A indutância teória mensurada a partir de um multímetro foi de $34 mH$ e a experimental, encontrada através do método gráfico, foi de $(20,30 \pm 1,33) mH$. Finalmente, a resistência teórica é de $100 \Omega$ e experimental, encontrada através do método gráfico, foi de $73,28 \pm 0,99 \Omega$. \\

Essa diferença de valores se da por conta de erros experimentais e um mal funcionamento dos cabos utilizados, onde esses erros podem ser diminuidos se houver uma troca para cabos sem mal contato, e osciloscópios mais novos. \\


\noindent \textbf{\Large Referências}
\begin{enumerate}
\item Toginho Filho,D. O., Laureto, E, Catálogo de Experimentos do Laboratório Integrado de Física Geral Departamento de Física - “Circuitos simples em corrente alternada Resistor, Capacitor e Indutor” - Universidade Estadual de Londrina, Março de 2009.

%backup-codes

%w****@gmail.com

%6459 7081
%2869 7029
%1100 1163

%g**

%bf9bd-0db9c
%a3df0-8a20c
%00470-6d921

%h****@uel.br

%2879 0028
%7863 5608
%5052 9089

%\item YOUNG, Hugh D.; FREEDMAN, Roger A., FISICA IV - ÓTICA E FÍSICA MODERNA, 12a ed. São Paulo, Addison Wesley, 2008.



%\item J.L. Duarte, C.R. Appoloni, A. Tannous, D.O. Toginho Filho, e F.V.D.Zapparoli, “Roteiros de Laboratório, Laboratório Física Geral IIB”, Universidade Estadual de Londrina, 2007. 

%\item Halliday D., Resnick, R., Walker, J., “Fundamentos de Física 4”, Livros Técnicos e Científicos Editora, 4a Edição, São Paulo, 1996. 

%\item Ueta, N; Vuolo, J. H. et al, Apostila de Laboratório de Física 4, “ Refração, Reflexão e Polarização”, USP, 1992

%\item Young e Freedman - “Física IV” - 12ª Edição - Capítulo 33 e 34  São Paulo: Addison Wesley, 2009. 

%\item  Toginho Filho, D. O., Zapparoli, F. V. D., Pantoja, J. C. S., - “Catálogo de Experimentos do Laboratório Integrado de Física Geral” - “Medições indiretas e propagação de erros” - Departamento de Física - Universidade Estadual de Londrina, Fevereiro de 2012.

\end{enumerate}
\end{multicols}
\end{document}

