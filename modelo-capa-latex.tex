\documentclass[a4paper, 12pt]{article}

\usepackage[portuges]{babel}
\usepackage[utf8]{inputenc}
\usepackage{amsmath}
%\mathcal{L/H} - L/H da lagrangeana/hamiltoniana 
\usepackage{indentfirst}
\usepackage{graphicx}
%\begin{figure}[!ht]
%\centering
%\includegraphics[width=2cm]{c:/ufba.jpg}
%\end{figure}
%Imagens dentro do ambiente multicols:
%\begingroup
%\centering
%\includegraphics[width=10cm]{h:/Will/Trabalhos_2019/tabela_1.jpg}
%\endgroup
\usepackage[a4paper,left=1.5cm,right=1.5cm,top=2.5cm,bottom=2cm]{geometry}
\usepackage{multicol}
%\begin{multicols}{2}
%\end{multicols}
\usepackage{cancel}
\usepackage{amsfonts}
\usepackage{amssymb}
\usepackage[pdftex]{hyperref}
%\url{link} 
%vetor: \vec{x}
%versor: \hat{e}
%produto interno de A e B: A\cdot B
%produto externo de A e B: A\wedge B
%bracket: \ | \rangle , dual: \langle \ |
%matriz: \begin{pmatrix} 1 & 4 & 0 \\ -2 & 4 & 3\end{pmatrix}
%complexo conjugado: \overline{z}
%chapéu do operador: \hat{M}
%R dos reais: \mathbb{R}
%derivação calculada no ponto x: derivação \Big\vert_{x}
%função com diferentes regras: V(x)=\left\{\begin{array}{ll}\infty , |x| > L/2\\0, |x| < L/2 \end{array}\right.
% aspas: `` ".
\begin{document}

\begin{titlepage}
	\begin{center}
        \begin{figure}[!ht]
        \centering
        \includegraphics[width=12cm]{d:/Will/Trabalhos/logo_uel.jpg}
        \end{figure}
        \begin{figure}[!ht]
        \centering
        \includegraphics[width=14cm]{d:/Will/Trabalhos/linha_verde.jpg}
        \end{figure}
		\large WILLIAM MAKOTO HAKAMADA 
		
	
		\vspace{20pt}
        \vspace{95pt}
        \vspace{50pt}
        
        \textbf{\LARGE{Relatório do experimento do Efeito Fotoelétrico}} \\
        
        \vspace{15pt}
        \vspace{95pt}
        \vspace{30pt}
		\vspace{3,5cm}
	\end{center}
	 \begin{figure}[!ht]
        \centering
        \includegraphics[width=14cm]{d:/Will/Trabalhos/linha_verde.jpg}
        \end{figure}
	\begin{center}
			 Londrina\\
		 2020
			\end{center}
\end{titlepage}

\begin{titlepage}

\begin{center}
\large WILLIAM MAKOTO HAKAMADA
 

        \vspace{20pt}
        \vspace{95pt}
        \vspace{50pt}
        
        \textbf{\LARGE{Relatório do experimento do Efeito Fotoelétrico}} \\
\end{center}

        \vspace{20pt}
        \vspace{95pt}
       
        
\begin{flushright}

   \begin{list}{}{
      \setlength{\leftmargin}{8cm}
      \setlength{\rightmargin}{0cm}
      \setlength{\labelwidth}{0pt}
      \setlength{\labelsep}{\leftmargin}}

      \item Relatório do experimento do Efeito Fotoelétrico apresentado à disciplina de Laboratório de Física Moderna da Universidade Estadual de Londrina.

      \begin{list}{}{
      \setlength{\leftmargin}{0cm}
      \setlength{\rightmargin}{0cm}
      \setlength{\labelwidth}{0pt}
      \setlength{\labelsep}{\leftmargin}}
			
            \item Orientador: Prof. Dr. Jair Scarminio \

      \end{list}
   \end{list}
\end{flushright}

        \vspace{20pt}
        \vspace{95pt}
        \vspace{50pt}
        
        	\begin{center}
			 Londrina\\
		 2020
			\end{center}
\end{titlepage}


\end{document}